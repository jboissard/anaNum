\chapter{Interpolation}
\section{Position du problème}
Soit $n \geq 0$ un nombre entier. Etant donnés $n + 1$ points distincts $x_0 , x_1, \dots,x_n$ 
et $n + 1$ valeurs $y_0, y_1,\dots, y_n$, on cherche un polynôme $p$ de degré $n$, tel que
\begin{eqnarray}
	p(x)=y_j&0\leq j\leq n
\end{eqnarray}
Dans le cas affirmatif, on note $p = \Pi_n$ et on appelle $\Pi_n$ le polynôme 
d’interpolation aux points $x_j$, $j = 0, \dots , n.$ 

Soit $f \in C^0(I)$ et $x_0 ,\dots , x_n \in I$ . Si comme valeurs $y_j$ on prend $y_j = f (x_j)$, $0 \leq j \leq n$, alors le polynôme d’interpolation $\Pi_n(x)$ est noté $\Pi_n f(x)$ et est appelé l’interpolant de $f$ aux points $x_0,\dots, x_n$.

\section{Base de Lagrange}
On considère les polynômes $\varphi_k, k = 0, \dots , n$ de degré $n$ tels que 
\begin{eqnarray*}
	\varphi_k (x_j) = \delta_{jk},& k, j = 0, \dots , n, 
\end{eqnarray*}
où $\delta_{jk}= 1$ si $j=k$ et $\delta_{jk}= 0$ si $j \neq k$. Explicitement, on a
\begin{eqnarray}
	\varphi_k(x)=\Pi_{j=0,j\neq k}^n\frac{(x-x_j)}{(x_k-x_j)}.
\end{eqnarray}
Le polynôme d'interpolation $Pi_n$ des valeurs $y_j$ aux points $x_j, j=0,\dots,n$, s'écrit
\begin{eqnarray}
	\Pi_n=\sum_{k=0}^{n}y_k\varphi_k(x),
\end{eqnarray}
car il vérifie $\Pi_n(x_j)=\sum_{k=0}^ny_k\varphi_k(x_j)=y_j$.

Par conséquent on aura
\begin{eqnarray}
	\Pi_nf(x)=\sum_{k=0}^nf(x_k)\varphi_k(x).
\end{eqnarray}

\section{Interpolation d'une fonction régulière}
\textbf{Théorème (Erreur d'interpolation)}
Soient $x_0,x_1,\dots,x_n, n+1$ noeuds équirépartis dans $I=[a,b]$ et soit $f\in C^{n+1}(I)$. Alors, on a
\begin{eqnarray}
	E_n(f)=\max_{x\in I}|f(x)-\Pi_nf(x)|\leq\frac{1}{4(n+1)}\left(\frac{b-a}{n}\right)^{n+1}\max_{x\in I}|f^{(n+1)}(x)|.
\end{eqnarray}


\section{Interpolation par intervalles}
Soient $x_0 = a < x_1 < \dots < x_N = b$ des points qui divisent l’intervalle $I = [a, b]$ 
dans une réunion d’intervalles $I_i= [x_i, x_{i+1}]$ de longueur $H$ où $H=\frac{b-a}{N}$.

Sur chaque sous-intervalle $I_i$ on interpole $f_{|I_i}$
par un polynôme de degré 1. Le polynôme par morceaux qu’on obtient est noté $\Pi_1^Hf(x)$, et on a:
\begin{eqnarray}
	\Pi_1^Hf(x)=f(x_i)+\frac{f(x_{i+1})-f(x_i)}{x_{i+1}-x_i}(x-x_i)&x\in I_i.
\end{eqnarray}

\textbf{Théorème} 
Si $f \in C^2(I)$, ($I = [x_0 , x_N]$) alors l'erreur est donnée par (pour $n\geq1$)
\begin{eqnarray}
	E_n^H(f)\leq\frac{H^{n+1}}{4(n+1)}\max_{x\in I}|f^{(n+1)}(x)|.
\end{eqnarray}
\section{Approximation au sens des moindres carrés}
\textbf{Définition} On appelle polynôme aux moindres carrés de degré $m$,
$\tilde f_m$ 
le polynôme de degré m tel que
\begin{eqnarray}
	\sum_{i=0}^n|y_i-\tilde f_m(x_i)|^2\leq\sum_{i=0}^{n}|y_i-p_m(x_i)|^2&\forall p_m(x)\in\mathbb P_m
\end{eqnarray}
Lorsque $y_i = f(x_i)$ ($f$ étant une fonction continue) alors 
$\tilde f_m$
est dit l’approximation de $f$ au sens des moindres carrés.

Autrement dit, le polynôme aux moindres carrés est le polynôme de degré $m$ 
qui, parmi tous les polynômes de degré $m$, minimise la distance des données. 
Si on note 
$\tilde f_m(x) = a_0 + a_1x + a_2x^2 + \dots + a_mx^m$, et on définit la fonction
\begin{eqnarray}
	\Phi(a_0,a_1,\dots,a_m)=\sum_{i=0}^n[y_i-(a_0+a_1x_i + a_2 x_i^2 + \dots + a_mx_i^m)]^2
\end{eqnarray}
alors les coefficients du polynôme aux moindres carrés peuvent être 
déterminés par les relations
\begin{eqnarray}
	\frac{\partial\Phi}{\partial\alpha_k}=0,&0\leq k\leq m
\end{eqnarray}
ce qui nous donne $m + 1$ relations linéaires entre les $a_k$.

En général, on observe que si pour calculer le polynôme interpolant aux moindres carrés 
$\tilde f_m(x)$
on impose les conditions d’interpolation
$\tilde f_m(x_i) = y_i$ 
pour $i = 0, \dots, n$, 
alors on trouve le système linéaire 
$Ba = \tilde y$, 
où $B$ est la matrice de dimension 
$(n + 1) \times (m + 1)$
\begin{eqnarray*}
	B=
	\begin{pmatrix}
		1 & x_0 & \cdots & x_0^m
		\\
		1 & x_1 & \cdots & x_1^m
		\\
		\vdots && & \vdots
		\\
		1 & x_n & \cdots & x_n^m
	\end{pmatrix}
\end{eqnarray*}
Puisque $m < n$ le système est surdéterminé, c’est-à-dire que le nombre de lignes est plus grand 
que le nombre de colonnes. Donc on ne peut pas résoudre ce système de façon classique, mais on 
doit le résoudre au sens des moindres carrés, en considérant:
\begin{eqnarray}
	B^TBa=B^T\tilde y.
\end{eqnarray}
De cette façon on trouve le système linéaire 
$Aa = y$ (avec $A = B^TB$ et $y = B^T\tilde y$), dit système 
d’équations normales. On peut montrer que les équations normales sont équivalentes au système.
\section{Interpolation par fonction splines}
Soient $a = x_0 < x_1 < \dots < x_n = b$ des points qui divisent l’intervalle $I = [a, b]$
dans une réunion d’intervalles $I_i = [x_i , x_{i+1}].$

\textbf{Définition.} On appelle spline cubique interpolant $f$ une fonction $s_3$ qui 
satisfait 
\begin{enumerate}
	\item $s_{3|_{I_i}}\in\mathbb P_3$ pour tout $i=0,\dots,n-1$, $\mathbb P_3$ étant l'ensemble des polynômes de degré 3,
	\item $s_3(x_i)=f(x_i)$ pour tout $i=0,\dots,n$,
	\item $s_3\in C^2([a,b])$.
\end{enumerate}

Cela revient à vérifier les conditions suivantes (on indique par 
$s_3(xi^-)$ la limite à gauche de $s_3$ au point $x_i$ 
et par 
$s_3(x_i^+)$ la limite à droite) :
\begin{eqnarray*}
s_3(x_i^-)=f(x_i)&1\leq i\leq n-1,\\
s_3(x_i^+)=f(x_i)&1\leq i\leq n-1,\\
s_3(x_0)=f(x_0),\\
s_3(x_n)=f(x_n),\\
s'_3(x_i^-)=s'_3(x_i^+)&1\leq i\leq n-1,\\
s''_3(x_i^-)=s''_3(x_i^+)&1\leq i\leq n-1,
\end{eqnarray*}
c’est-à-dire $2(n-1) + 2 + 2(n- 1) = 4n-2$ conditions. 

Il faut trouver $4n$ inconnues (qui sont les 4 coefficients de chacune des $n$ 
restrictions $s_{3|_{I_i}}$, $i=0,\dots,n-1$) et on dispose de $4n-2$ relations. 
On rajoute alors 2 conditions supplémentaires à vérifier.
\begin{itemize}
	\item Si l'on impose
	\begin{eqnarray*}
		s''_3(x_0^+)=0\text{ et }s''_3(x_n^-)=0
	\end{eqnarray*}
	alors la spline $s_3$ est complètement déterminée et s’appelle 
	\emph{spline naturelle}.
	
	\item Une autre possibilité est d’imposer la continuité des dérivées troisièmes 
	dans les noeuds $x_2$ et $x_{n-1}$ , c’est à dire:
	\begin{eqnarray*}
		s'''_3(x_2^-)=s'''_3(x_2^+)\text{ et }s'''_3(x_{n-1}^-)=s'''_3(x_{n-1}^+)
	\end{eqnarray*}
	dans ce cas elle s'appelle \emph{not-a-knot}.
\end{itemize}