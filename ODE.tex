\chapter{Equations différentielles Ordinaires (ODE)}	
\section{Problème général de Cauchy}

\begin{eqnarray}
	\begin{cases}
		y'(t)=f(t,y(t))&t>0
		\\
		y(0)=y_0
	\end{cases}
\end{eqnarray}
Un problème de Cauchy peut être 
\begin{itemize}
	\item linéaire
	\item non-linéaire
\end{itemize}

\section{Dérivées numériques}
Différence finie \emph{progressive}
\begin{eqnarray}
	(Dy)_i^P=\frac{y(t_{i+1})-y(t_i)}{h}&i=0,\dots,n-1
\end{eqnarray}
Différence finie \emph{rétrograde}
\begin{eqnarray}
	(Dy)_i^R=\frac{y(t_{i})-y(t_{i-1})}{h}&i=0,\dots,n	
\end{eqnarray}
Différence finie \emph{centrée}
\begin{eqnarray}
	(Dy)_i^C=\frac{y(t_{i+1})-y(t_{i-1})}{2h}&i=0,\dots,n-1	
\end{eqnarray}


\subsection{Erreur commise}
Il existe pour tout $t$, un $\eta$ entre $t_i$ et $t$, un développement de Taylor
\begin{eqnarray}
	y(t)=y(t_i)+y'(t_i)(t-t_i)+\frac{y''(\eta)}{2}(t-t_i)^2
\end{eqnarray}
On obtient les \emph{estimations} suivantes
\begin{enumerate}
	\item \begin{eqnarray}
		|y'(t_i)-(Dy)_i^P|\leq Ch,&\text{où}&C=\frac{1}{2}\max_{t\in[t_i,t_{i+1}]}|y''(t)|
	\end{eqnarray}
	\item \begin{eqnarray}
		|y'(t_i)-(Dy)_i^R|\leq Ch,&\text{où}&C=\frac{1}{2}\max_{t\in[t_i,t_{i+1}]}|y''(t)|
	\end{eqnarray}
	\item \begin{eqnarray}
		|y'(t_i)-(Dy)_i^C|\leq Ch^2,&\text{où}&C=\frac{1}{6}\max_{t\in[t_i,t_{i+1}]}|y'''(t)|
	\end{eqnarray}
\end{enumerate}

La différence $\tau(h)=|y'(t_i)-(Dy)_i^P|$ est appelée \textbf{erreur de troncature en $t_i$}

On dira que $\tau$ est d'ordre $p>0$, si: $\tau(h)\leq Ch^P$, pour une constante $C>0$, on obtient que 
\begin{enumerate}
	\item ordre $p=1$
	\item ordre $p=1$
	\item ordre $p=2$
\end{enumerate}

\section{Méthodes des différences finies}
\begin{enumerate}
	\item Euler progressif
	\begin{eqnarray}
		\begin{cases}
			\frac{u_{n+1}-u_n}{h}=f(t_n,u_n)&n=0,1,2,\dots
			\\
			u_0=y_0
		\end{cases}
	\end{eqnarray}
	Schéma explicite
	\item Euler rétrograde
	\begin{eqnarray}
	\begin{cases}
		\frac{u_{n+1}-u_n}{h}=f(t_{n+1},u_{n+1})&n=0,1,2,\dots
		\\
		u_0=y_0
	\end{cases}
	\end{eqnarray}
	Schéma implicite
\end{enumerate}

\section{Conditions de stabilité}
Le choix du pas de temps $h$ n’est pas arbitraire et doit être choisi judicieusement. On verra plus tard que si $h$ est trop petit, des problèmes de stabilité peuvent survenir.

\subsection{Propriétés de stabilité (absolue)}
Soit $\lambda<0$, on considère le problème suivant
\begin{eqnarray}
	\begin{cases}
		y'(t)=\lambda y(t)& \forall\mathbb{R}_+
		\\
		y
	\end{cases}
\end{eqnarray}
Dont la solution est $y(t)=y_0e^{-\lambda t}$ et $\lim_{t\rightarrow\infty}y(t)=0$

Un schéma de dérivation est dit \textbf{absolument stable} si $\lim_{n\rightarrow\infty}u_n=0$

\begin{enumerate}
	\item Schéma d'Euler progressif
	\begin{eqnarray}
		u_{n+1}=(1+\lambda h)u_n&u_n=(1+\lambda h)^ny_0 &\forall n\geq 0
	\end{eqnarray}
	La condition de stabilité est $|1+\lambda h|<1$ d'où $h<\frac{2}{|\lambda|}$
	\item Schéma d'Euler rétrograde
	\begin{eqnarray}
		u_{n+1}=(1+\lambda h)u_n&u_n=(1+\lambda h)^ny_0 &\forall n\geq 0
	\end{eqnarray}
	comme $\lim_{n\rightarrow\infty}u_n=0$ ce schéma est \textbf{inconditionellement} stable
\end{enumerate}

\subsection{Stabilité absolue contrôle les perturbations}
Considérons le modèle généralisé suivant:
\begin{eqnarray}
	\begin{cases}
		y'(t)=\lambda(t)y(t)+r(t)&t\in(0,\infty)
		\\
		y(0)=1
	\end{cases}
\end{eqnarray}
soient $r$ et $\lambda$ des fonctions continues:

$\lambda_{max}\leq\lambda(t)\leq\lambda_{min}$ où $0<\lambda_{min}\leq\lambda_{max}<\infty$

\subsection{Convergence d'Euler progressif}
La méthode est convergente si $\forall n=1,\dots,N_h$:
\begin{eqnarray}
	|u_n-y(t_n)|\leq C(h)&\text{où } C(h)\rightarrow 0 \text{ when } h\rightarrow 0
\end{eqnarray}
Si en plus $\exists p>0$, tel que $C(h)=O(h^P)$ alors on parle de méthode \textbf{convergente d'ordre $p$}

\section{Récapitulatif}

\begin{tabular}{| c || c | c | c | c |}
	\hline
	\emph{Méthode} & \emph{Explicite/Implicite} & \emph{Pas} & \emph{Stabilité} & \emph{Ordre}
	\\
	\hline
	Euler progressif & Explicite & 1 & Conditionellement & 1
	\\
	Euler rétrograde & Implicite & 1 & Inconditionellement & 1
	\\
	Crank Nicholson  & Implicite & 1 & Inconditionellement & 2
	\\
	Heun & Explicite & 1 & Conditionellement & 2
	\\
	Euler modifié & Explicite & 1 & Conditionellement & 2
	\\
	\hline
\end{tabular}

\subsection{Système d'ODE}
\begin{eqnarray}
	\mathbf{y'}(t)=A\mathbf{y}(t)+\mathbf{b}(t)&t>0,
	\\
	\mathbf{y}(0)=\mathbf{y}_0
\end{eqnarray}
$A\in\mathbb R^{pxp}$ et $\mathbf{b}(t)\in\mathbb R^p$ où l'on suppose que $\mathbf{A}$ possède $p$ valeurs propres distinctes $\lambda_j, j=1,\dots,p$

La méthode d'Euler progressif devient
\begin{eqnarray}
	\begin{cases}
		\frac{\mathbf{u}_{n+1}-\mathbf{u}_n}{h}=A\mathbf{u}_n+\mathbf{b}_n&n=0,1,2,\dots
		\\
		\mathbf{u}_0=\mathbf{y}_0
	\end{cases}
\end{eqnarray}

La méthode d'Euler rétrograde devient
\begin{eqnarray}
	\begin{cases}
		\frac{\mathbf{u}_{n+1}-\mathbf{u}_n}{h}=A\mathbf{u}_{n+1}+\mathbf{b}_{n+1}&n=0,1,2,\dots
		\\
		\mathbf{u}_0=\mathbf{y}_0
	\end{cases}
\end{eqnarray}

\subsection{Stabilité d'un système d'ODE}
La méthode d'Euler est \textbf{\emph{stable}} pour autant que la condition suivante soit vérifiée
\begin{eqnarray}
	h<\frac{2}{\max_{j=1,\dots,p}|\lambda_j|}=\frac{2}{\rho(A)}
\end{eqnarray}
où $\rho(A)$ est le rayon spectral de $A$.

La méthode d'Euler rétrograde est \textbf{\emph{inconditionellement stable}}.
