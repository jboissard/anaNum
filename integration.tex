\chapter{Intégration numérique}
\section{Formule d'intégrations simples}
Soit $f :[a,b] \rightarrow\mathbb R$ une fonction continue donnée sur un intervalle $[a,b]$. On désire calculer numériquement la quantité : 
\begin{eqnarray}
	I(f)=\int_a^b f(x)\,dx	
\end{eqnarray}
On considère les formules d’intégrations simples suivantes : 
\begin{enumerate}
	\item Formule du rectangle :
	\begin{eqnarray}
		I_{mp}(f)=(b-a)f(\frac{a+b}{2})
	\end{eqnarray}
	\item Formule du trapèze:
	\begin{eqnarray}
		I_t(f)=(b-a)\frac{f(a)+f(b)}{2}
	\end{eqnarray}
	\item Formule de Simpson:
	\begin{eqnarray}
		I_s(f)=\frac{b-a}{6}\left[f(a)+4f(\frac{a+b}{2})+f(b)\right]
	\end{eqnarray}
\end{enumerate}

\section{Formules d'intégrations composites}
On va considerer les $M$ sous-intervalles $I_k = [x_{k-1},x_k]$, $k=1,\dots,M$, 
où 
$x_k = a + kH$ et $H = (b - a)/M$ . Comme on a 
\begin{eqnarray}
	I(f)=\sum_{K=1}^M\int_{I_k}f(x)\,dx
\end{eqnarray}
sur chaque sous-intervalle $I_k$ on peut calculer une approximation de l’intégrale 
exacte de $f$ avec l’intégrale d’un polynôme $f$ approchant $\overline f$ sur $I_k$, c.-à.d.: 
\begin{eqnarray}
	I (f ) \text{ approchée par } \sum_{K=1}^M\int_{I_k}f(x)\,dx=\int_a^b\Pi_n^Hf(x)\,dx.
\end{eqnarray}

\subsection{Formules d'intégration composite du rectangle}
Cette formule est obtenue en remplaçant, sur chaque sous-intervalle $I_k$, la 
fonction $f$ par un polynôme constant $\Pi_of$ égal à la valeur de $f$ au milieu de $I_k$: on obtient la formule composite du rectangle (ou du 
point milieu) 
\begin{eqnarray}
	I_{pm}^c(f)=H\sum_{k=1}^Mf(\overline x_k),
\end{eqnarray}
où $\overline x_k=\frac{x_{k-1}+x_k}{2}$

\subsection{Formules d'intégration composite du trapèze}

Si sur chaque intervalle $I_k$ on remplace $f$ par le polynôme d’interpolation 
$\overline f = \Pi_1 f$ de degré 1 aux noeuds $x_{k-1}$ et $x_k$, on obtient la formule composite du 
trapèze: 
\begin{eqnarray}
	I_t^c(f)=\frac{H}{2}\sum_{k=1}^{M}\left[f(x_k)+f(x_{k-1})\right]=\frac{H}{2}[f(a)+f(b)]+H\sum_{k=1}^{M-1}f(x_k).
\end{eqnarray}

\subsection{Formules d'intégration composite de Simpson}

La formule de Simpson est obtenue en remplaçant $f$ par son polynôme 
interpolant composite $\overline f = \Pi_2^H$ de degré 2 entre les noeuds $x_k$. 
En particulier, $f$ est une fonction continue par morceaux qui sur chaque sous-intervalle $I_k$ est obtenue comme le polynôme interpolant $f$ aux noeuds $x_{k-1}, \overline x_k=\frac{x_{k-1}+x_k}{2}$ et $x_k$.

On obtient donc la formule composite de Simpson: 
\begin{eqnarray}
	I_s^c(f)=\frac{H}{6}\sum_{k=1}^M\left[f(x_{k-1})+4f(\overline x_k)+f(x_k)\right]
\end{eqnarray}
\section{Erreur d'intégration numérique}
\begin{enumerate}
	\item Formule composite du rectangle, si $f$ est dans $C^2([a, b])$, alors
	\begin{eqnarray}
		|I(f)-I_{pm}^c(f)|\leq\frac{b-a}{24}H^2\max_{x\in[a,b]}|f''(x)|
	\end{eqnarray}
	\item Formule composite du trapèze, si $f$ est dans $C^2([a, b])$, alors
	\begin{eqnarray}
		|I(f)-I_{t}^c(f)|\leq\frac{b-a}{12}H^2\max_{x\in[a,b]}|f''(x)|
	\end{eqnarray}
	\item Formule composite de Simpson, si $f$ est dans $C^4([a, b])$, alors
	\begin{eqnarray}
		|I(f)-I_{s}^c(f)|\leq\frac{b-a}{180\cdot16}H^4\max_{x\in[a,b]}|f^{(4)}(x)|
	\end{eqnarray}

\end{enumerate}
Les formules ci-dessus sont des cas particuliers de la \textbf{formule générale de quadrature}
\begin{eqnarray}
	I_{approx}(f)=\sum_{k=0}^n\alpha_kf(x_k)
\end{eqnarray}
où $x_k$ sont les noeuds de la formule de quadrature, et $\alpha_k$ sont les poids (voir la table 
suivante).
\begin{center}
	

\begin{tabular}{| c | c | c | c |}
	\hline
	\emph{Formule} & \emph{Dg. Exact} & $\mathit{x_k}$ & $\mathit{\alpha_k}$
	\\
	\hline
	Rectangle & 1 & $x_0=\frac{1}{2}(a+b)$ & $\alpha_0=b-a$
	\\
	\hline
	Trapèze & 1 & $x_0=a, x_1=b$ & $\alpha_0=\alpha_1=\frac{1}{2}(b-a)$
	\\
	\hline
	Simpson & 3 & $x_0=a, x_1=\frac{1}{2}(a+b), x_2=b$ & $\alpha_0=\alpha_2=\frac{1}{6}(b-a),\alpha_1=\frac{2}{3}(b-a)$\\
	\hline
\end{tabular}
\begin{tabular}{| c || c | c |}
	\hline
	\emph{Formule composite} & \emph{Deg. exact} & \emph{Ordre par rapport à H}\\
	\hline
	Rectangle & 1 & 2
	\\
	\hline
	Trapèze & 1 & 2
	\\
	\hline
	Simpson & 2 & 4\\
	\hline
	
\end{tabular}
\end{center}